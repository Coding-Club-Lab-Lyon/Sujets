\documentclass[a4paper, 12pt]{article}

\usepackage{/template/configs/base}

\begin{document}

\Init{Pydiluve}{Juste avant la piscine}{290621}{1.0}
\InitInfo{pydiluve}{Bash, Python3}{turtle}

\Hypersetup

\Cover
\Intro{
    
    {\comfortaa
        Il faisait chaud, les vacances approchaient, les bars et restaurants rouvraient
        la période la plus dure de l'année est maintenant derrière nous.
        Tu l'as trouvé, tu as patienté en faisant la queue tu as même dû lacher
        un gros billet pour rentrer mais il reste une dernière chose à faire avant
        d'entrer en Piscine,
        \newline
        \underline{passer par la Pédiluve}.

        \ImageCenter{pediluve_obligatoire.jpg}{200}

        La municipalité d'Epitech vous a construit une piscine de qualité dans laquelle
        vous allez pouvoir faire du sale. Mais remettons les choses à plats. Avant de
        se jeter en bombe, il faut s'assurer que l'on a bien les pieds propres.
        Ce serait bête d'attirer microbes et moustiques.

    }
}

\section{Pied Droit : Bash}

{\comfortaa
    \ImageCenter{bash_logo.jpg}{100}

    Bash pour Bourne-Again Shell est un interpreteur de commandes très connu et très utilisé sur Linux. Même si aujourd'hui
    certains de ces petits frères tels que ZSH ou encore Fish tendent à le remplacer, il reste important tant
    son arrivée a standardisé la façon de communiquer avec son environnement Linux.
    \newline \newline
    On tape une commande et Bash va exécuter pour nous le logiciel correspondant
    (Tu comprendras tout ça une fois dans le grand bain)

    \Terminal{
        \$ echo "Hello World" \\
        Hello World
    }
    \vspace{0.6cm}

    Tu seras invité à chercher par toi même les outils et commandes
    qui te serviront à bien laver ton pied droit. Ceci dit nous allons
    quand même
    \newline
    exceptionellement t'en indiquer quelques uns et encore
    on ne sera pas complet.

    \Hint{Google est ton ami !}
    
    \begin{tabular}{|p{2cm}|p{14cm}|} 
        \hline
        ls & lister les fichiers présents à ton emplacement \\
        \hline
        cd & se déplacer jusqu'à un dossier \\
        \hline
        pwd & savoir où te situe \\
        \hline
        chmod & gérer les permissions d'un fichier \\
        \hline
        clear & remettre l'affichage à 0 \\
        \hline
        sudo & qui s'accompagne d'une commande, l'exécute avec les droits administrateurs \\
        \hline
        touch & créer un fichier \\
        \hline
        mkdir & créer un dossier \\
        \hline
        rm & supprimer un fichier \\
        \hline
        rmdir & supprimer un dossier \\
        \hline
        mv & déplacer un fichier ou un dossier \\
        \hline
    \end{tabular}

    Il faut savoir que lorsque qu'une commande est exécutée, elle peut être amenée à nous afficher
    des informations. Mais avant tout, à la fin de son exécution elle nous renvoie un statut
    sous la forme d'un nombre entier positif. C'est ce nombre qui nous permet directement de
    vérifier qu'une commande à été exécutée avec succès (si son statut est égal à 0).

    \subsection{getStatus}

    Lors de ton passage par la pédiluve, on va devoir vérifier l'état des pieds de tous les
    aventuriers. On va donc créer de quoi effectuer ces verifications.
    \newline \newline
    On va procéder de la manière suivante :
    \newline \newline
    On récupère le statut de la dernière commande. Si la commande
    s'est exécutée avec succès alors on affiche "Les pieds sont propres !"
    S'il y a eu un problème on affiche "T'as de ces ieps chacal..."
    \newline \newline

    Exemple :

    \Terminal{
        ls                          \\
        ticket                      \\
        \$> ./getStatus                   \\
        Les pieds sont propres !    \\
    }

    Ou encore :

    \Terminal{
        ls dossier-qui-nexiste-pas                                              \\
        ls: cannot access 'dossier-qui-nexiste-pas': No such file or directory  \\
        \$> ./getStatus                                                               \\
        T'as de ces ieps chacal...                                              \\
    }

    \Warning{N'oublie pas de régulièrement pousser ton travail sur Git ! De mauvaises surprises peuvent t'attendre et causer ta noyade}

    \subsection{countCharo}

    Les pieds des aventuries semblent aussi propres que le crâne de Mr Propre.
    On dirait que ton outil fonctionne bien. Alors on va l'améliorer en règlant
    un autre problème !
    \newline \newline
    Dans notre piscine se rendent aussi certaines jolies demoiselles qui viennent pricipalement
    d'ISEG City. Et nous avons pensé à elles en prenant soin de leur mettre à disposition d'autres vestiaires
    afin de respecter leur intimité. Or on peut régulièrement apercevoir des aventuriers d'Epitech
    entrer et sortir dans les mauvais vestiaires.
    \newline \newline
    On les appelle les \underline{Charos}
    \newline \newline
    Afin de les arrêter, il va falloir scanner les vestiaires "Epitech" et "ISEG" et vérifier qui s'y trouve.
    \newline
    \begin{itemize}
        \item Toutes les personnes dont le prénom se termine par la lettre 'a' sont des aventuriers d'Epitech
        \item Ceux dont le prénom se termine par la lettre 'z' sont des jolies demoiselles d'ISEG City
        \item Ceux ne correspondant à aucun des 2 critères sont des instrus et il faut appeler le videur pour les chasser.
    \end{itemize}

    \vspace{0.6cm}

    Il faut que tu compte le nombre d'aventuriers d'Epitech présents dans le mauvais vestiaire et que
    tu renvoie ce nombre en tant que statut.
    \newline \newline
    Dans ce premier exemple, le statut doit être égal à 0

    \Terminal{
        ls                                              \\
        Epitech/ ISEG/                                  \\
        \$> ls Epitech/                                 \\
        Grisha Koba Luca                                \\
        \$> ls ISEG/                                    \\
        Mariz Fatimaz Zazouz                            \\
        \$> ./countCharo                                \\
    }
    
    \vspace{4cm}
    Dans ce deuxième exemple, le statut doit être égal à 2 (les pauvres demoiselles)
    
    \Terminal{
        ls Epitech/ ISEG/                               \\
        Epitech/: Grisha                                \\
        ISEG/: Mariz Fatimaz Zazouz Koba Luca           \\
        \$> ./countCharo                                \\
    }

    Dans ce troisième exemple, le statut doit être égal à 1 et certains participants doivent être
    renvoyés
    
    \Terminal{
        ls -R                                           \\
        Epitech: Grisha Luca Mehdi                      \\
        ISEG: Mariz Fatimaz Zazouz Koba Jose            \\
        \$> ./countCharo                                \\
        \$> ls -R                                       \\
        Epitech: Grisha Luca                            \\
        ISEG: Mariz Fatimaz Zazouz Koba                 \\
    }
}

\section{Pied Gauche : Python}

{\comfortaa

}

\end{document}