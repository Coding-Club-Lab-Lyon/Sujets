\documentclass[a4paper, 12pt]{article}

\usepackage{/template/configs/base}

\begin{document}

\Init{Pydiluve}{Juste avant la piscine}{290621}{1.0}
\InitInfo{pydiluve}{Bash, Python3}{turtle}

\Hypersetup

\Cover
\Intro{
    
    {\comfortaa
        Il faisait chaud, les vacances approchaient, les bars et restaurants rouvraient
        la période la plus dure de l'année est maintenant derrière nous.
        Tu l'as trouvé, tu as patienté en faisant la queue tu as même dû lacher
        un gros billet pour rentrer mais il reste une dernière chose à faire avant
        d'entrer en Piscine,
        \newline
        \underline{passer par la Pédiluve}.

        \ImageCenter{pediluve_obligatoire.jpg}{200}

        La municipalité d'Epitech vous a construit une piscine de qualité dans laquelle
        vous allez pouvoir faire du sale, mais remettons les choses à plats:
        \newline
        Avant de se jeter en bombe, il faut s'assurer que l'on a bien les pieds propres.
        Ce serait bête d'attirer microbes et moustiques.
        \newline \newline
        Aujourd'hui nous allons apprendre à utiliser un shell Linux et à développer des
        algorithmes complexes en Python car il est très important de savoir les 2
        pour bien gérer une Piscine.
        \newline \newline
        Tu ne vois pas de rapport ? 
        Et bien crois moi tu vas le voir arriver de très loin.
    }
}

\section{Pied Droit : Bash}

{\comfortaa
    \ImageCenter{bash_logo.jpg}{100}

    Bash pour Bourne-Again Shell est un interpreteur de commandes très connu et très utilisé sur Linux. Même si aujourd'hui
    certains de ces petits frères tels que ZSH ou encore Fish tendent à le remplacer, il reste important tant
    son arrivée a standardisé la façon de communiquer avec son environnement Linux.
    \newline \newline
    On tape une commande et Bash va exécuter pour nous le logiciel correspondant
    (Tu comprendras tout ça une fois dans le grand bain)

    \Terminal{
        echo "Hello World" \\
        Hello World
    }
    \vspace{0.6cm}

    Tu seras invité à chercher par toi même les outils et commandes
    qui te serviront à bien laver ton pied droit. Ceci dit nous allons
    quand même
    \newline
    exceptionellement t'en indiquer quelques uns et encore
    on ne sera pas complet.

    \Hint{Google est ton ami !}
    
    \begin{tabular}{|p{2cm}|p{14cm}|} 
        \hline
        ls & lister les fichiers présents à ton emplacement \\
        \hline
        cd & se déplacer jusqu'à un dossier \\
        \hline
        pwd & savoir où te situe \\
        \hline
        chmod & gérer les permissions d'un fichier \\
        \hline
        clear & remettre l'affichage à 0 \\
        \hline
        sudo & qui s'accompagne d'une commande, l'exécute avec les droits administrateurs \\
        \hline
        touch & créer un fichier \\
        \hline
        mkdir & créer un dossier \\
        \hline
        rm & supprimer un fichier \\
        \hline
        rmdir & supprimer un dossier \\
        \hline
        mv & déplacer ou rennomer un fichier ou un dossier \\
        \hline
    \end{tabular}

    Ces commandes sont en réalité des programmes eux aussi. Donc faire ça:

    \Terminal{
        ls                          \\
        fichier-lol dossier-omg/
    }
    Revient à faire ça :
    \Terminal{
        /bin/./ls                   \\
        fichier-lol dossier-omg/
    }
    \vspace{0.6cm}

    Sans vraiment rentrer dans les détails (afin que tu le fasse par toi-même),
    il faut comprendre que toutes les commandes basiques sont des programmes placés dans des
    dossiers spécifiques où Bash ira chercher en priorité ce que tu lui demande de faire.
    \newline \newline
    Les scripts que tu t'apprête à coder ne seront pas (encore) dans ces dossiers spécifiques.
    Par conséquent tu devra mettre "./" (un point et un slash) devant le chemin vers ton programme :

    \Terminal{
        ls                          \\
        discord fortnite valorant   \\
        \$> ./discord
    }

    Normalement tout programme (fichier ayant les droits d'exécution) trouvé via "ls" est censé s'afficher en vert
    \newline \newline
    
    \Hint{
        Si tu es dépourvu de couleur, tu peux te repérer en faisant "ls -l".
        Il faut alors que tu vérifies le 4e caractère au début de la ligne de ton fichier
        soit bien 'x' et non pas '-'.
    }
    \vspace{3cm}
    
    \Terminal{
        ls -l                                               \\
        total 3                                             \\
        -rwxr-xr-x 1 toi toi  33156 Jun 29 00:38 fortnite   \\
        -rwxr-xr-x 1 toi toi    137 Jun 29 00:38 discord    \\
        -rw-r--r-- 1 toi toi   7158 Jun 28 21:54 valorant
    }

    Du coup, on peut constater que "valorant" risque de ne pas s'exécuter.
    \newline
    Pour régler ce problème on doit lui donner les droits d'exécution !
    
    \vspace{0.6cm}
    \Hint{Et si on demandait à Google comment faire ça ?}
    
    \vspace{0.6cm}

    \ImageCenter{google_research.jpg}{150}

    \vspace{2cm}
    Il y a une dernière chose à savoir : lorsque qu'un programme est exécuté, il peut être amené à nous afficher
    des informations directement sur le terminal mais à la fin de son exécution il nous renvoie systématiquement un statut
    sous la forme d'un nombre entier positif. C'est ce nombre qui nous permet de
    vérifier qu'un programme à été exécutée avec succès (si son statut est égal à 0).
    \vspace{4cm}

    \subsection{getStatus}

    Lors de ton passage par la pédiluve, on va devoir vérifier l'état des pieds de tous les
    aventuriers. On va donc créer de quoi effectuer ces verifications.
    \newline \newline
    On va procéder de la manière suivante :
    \newline \newline
    On récupère le statut de la dernière commande. Si la commande
    s'est exécutée avec succès alors on affiche "Les pieds sont propres !"
    S'il y a eu un problème on affiche "T'as de ces ieps chacal..."
    \newline \newline

    Exemple :

    \Terminal{
        ls                          \\
        ticket                      \\
        \$> ./getStatus                   \\
        Les pieds sont propres !    \\
    }

    Ou encore :

    \Terminal{
        ls dossier-qui-nexiste-pas                                              \\
        ls: cannot access 'dossier-qui-nexiste-pas': No such file or directory  \\
        \$> ./getStatus                                                               \\
        T'as de ces ieps chacal...                                              \\
    }
    
    \Hint{T'as déjà entendu parler des Shebangs}

    \Warning{N'oublie pas de régulièrement pousser ton travail sur Git ! De mauvaises surprises peuvent t'attendre et causer ta noyade}

    \vspace{3cm}

    \subsection{countCharo}

    Les pieds des aventuries semblent aussi propres que le crâne de Mr Propre.
    On dirait que ton outil fonctionne bien. Alors on va l'améliorer en règlant
    un autre problème !
    \newline \newline
    Dans notre piscine se rendent aussi certaines jolies demoiselles qui viennent pricipalement
    d'ISEG City. Et nous avons pensé à elles en prenant soin de leur mettre à disposition d'autres vestiaires
    afin de respecter leur intimité. Or on peut régulièrement apercevoir des aventuriers d'Epitech
    entrer et sortir dans les mauvais vestiaires.
    \newline \newline
    On les appelle les \underline{Charos}
    \newline \newline
    Afin de les arrêter, il va falloir scanner les vestiaires "Epitech" et "ISEG" et vérifier qui s'y trouve.
    \newline
    \begin{itemize}
        \item Toutes les personnes dont le prénom se termine par la lettre 'a' sont des aventuriers d'Epitech
        \item Ceux dont le prénom se termine par la lettre 'z' sont des jolies demoiselles d'ISEG City
        \item Ceux ne correspondant à aucun des 2 critères sont des instrus et il faut appeler le videur pour les chasser.
    \end{itemize}

    \vspace{0.6cm}

    Il faut que tu compte le nombre d'aventuriers d'Epitech présents dans le mauvais vestiaire et que
    tu renvoie ce nombre en tant que statut.
    \newline \newline
    Dans ce premier exemple, le statut doit être égal à 0

    \Terminal{
        ls                                              \\
        Epitech/ ISEG/                                  \\
        \$> ls Epitech/                                 \\
        Grisha Koba Luca                                \\
        \$> ls ISEG/                                    \\
        Mariz Fatimaz Zazouz                            \\
        \$> ./countCharo
    }
    
    \vspace{4cm}
    Dans ce deuxième exemple, le statut doit être égal à 2 (les pauvres demoiselles)
    
    \Terminal{
        ls Epitech/ ISEG/                               \\
        Epitech/: Grisha                                \\
        ISEG/: Mariz Fatimaz Zazouz Koba Luca           \\
        \$> ./countCharo
    }

    Dans ce troisième exemple, le statut doit être égal à 1 et certains participants doivent être
    renvoyés
    
    \Terminal{
        ls -R                                           \\
        Epitech: Grisha Luca Mehdi                      \\
        ISEG: Mariz Fatimaz Zazouz Koba Jose            \\
        \$> ./countCharo                                \\
        \$> ls -R                                       \\
        Epitech: Grisha Luca                            \\
        ISEG: Mariz Fatimaz Zazouz Koba
    }

    \subsection{cloackroomCleaner}

    \subsubsection{Chacun dans ces propres vestiaires}

    Ce problème n'a pas été simple à régler mais tu y es partiellement parvenu bravo !
    Il est temps de finir le travail en beauté en redirigeant tous nos clients
    dans les bons vestiaires, toujours en renvoyant les intrus.
    \newline \newline
    Le staut devra être 0 dorénavant.
    \newline \newline
    
    Exemple :
    \Terminal{
        ls -R                                           \\
        Epitech: Grisha Luca                            \\
        ISEG: Mariz Fatimaz Zazouz Koba                 \\
        \$> ./cloackroomCleaner                         \\
        \$> ls -R                                       \\
        Epitech: Grisha Luca Koba                       \\
        ISEG: Mariz Fatimaz Zazouz
    }
    \vspace{1cm}
    
    Ou encore, avec un intru en supplément :
    \Terminal{
        ls -R                                           \\
        Epitech: Mariz Fatimaz Zazouz Giroud            \\
        ISEG: Grisha Luca Koba                          \\
        \$> ./cloackroomCleaner                         \\
        \$> ls -R                                       \\
        Epitech: Grisha Luca Koba                       \\
        ISEG: Mariz Fatimaz Zazouz
    }
    
    \vspace{2cm}
    
    \subsubsection{Chasser les Méga-Charos}

    Nous n'avons jusque là rencontré que des petits Charos. Il est maintenant
    temps de se confronter aux VRAIS. Ceux-là sont des professionels
    et "n'ont pas le temps de niaiser" comme ils disent. Ils représentent
    une menace importante pour notre Piscine et doivent être chassés au même titre
    que les intrus.
    \newline \newline
    Ouvre bien les yeux car ils ont tendances à se cacher dans les vestiaires.

    \Hint{"ls" n'a pas encore dit son dernier mot}

    \vspace{6cm}

    \ImageCenter{bravo.png}{150}

    Félicitations !
    Tu as réussi à faire régner l'ordre dans notre Piscine le tout en nettoyant ton pied droit.
    Il nous reste encore du chemin à faire mais cette fois-ci avec le 2e pied.

    \vspace{2cm}
    
    \subsection{Bonus}
    Juste avant de passer à la suite je te propose de faire de tes programmes des commandes
    Si tu te rappelle encore de ce que je te disais au début par rapport aux programmes
    tu dois sûrement te rappeler que les commandes sont trouvées car présentes dans des dossiers
    spécifiques. Nous allons donc rajouter parmis les dossier spécifiques le dossier actuel dans lequel tu
    te situe. Ainsi tes commandes pourront être lancées depuis n'importe où.
    \newline \newline
    On va ajouter ton dossier dans les dossiers prioritaires puis redémarrer ton shell

    \Terminal{
        echo "export PATH=\$(pwd):\$PATH" >> ~/.bashrc    \\
        source ~/.bashrc
    }
    \vspace{4cm}
}


\section{Pied Gauche : Python}

{\comfortaa
    Je vois que tu es un dûr qui n'a pas peur d'affronter les problèmes.
    La complexité va monter d'un cran mais je te rassure tes moyens aussi.
    \newline \newline
    Maintenant que tu maîtrise Bash je te propose de passer à Python.

    Avant même de commencer à coder je dois te parler des paquets :
    \newline \newline
    Il s'agit de logiciels que tu peux installer depuis des répertoires distants à l'aide
    de Gestionnaires de Paquets. La distribution Linux sur laquelle tu te situe en est
    forcément pourvoue d'une. Nous partons du principe que tu es sur Ubuntu et dans ce
    cas le package manager est "apt". Tu dois te demander pourquoi je te parle de ça ?
    \newline
    C'est parce que sur Linux (du moins certaines de ces distributions) tu commences avec rien
    et tu dois par conséquent tout installer, y compris Python par exemple.

    Si tu es sur Ubuntu Python est normalement déjà installé et ça peut se vérifier :

    \Terminal{
        python3 -\--version
    }

    Si tu ne possède pas Python alors tu peut l'installer via ton Gestionnaire de Paquets
    Commence à mettre à jour tes paquets et installe python3

    \Terminal{
        sudo apt update                 \\
        \$> sudo apt upgrade            \\
        \$> sudo apt install python3
    }

    \subsection{octogon}

    \ImageCenter{assets/one.png}{100}
    \ImageCenter{assets/two.png}{300}

    Il sera important de tester le binaire fournis afin de comprendre la logique.

    \Warning{La fonction suivante vous sera demandée !}
    \Terminal{def printcircle(radius):}
    Cette fonction devra calculer et afficher chaques parties du cercle.
    \Warning{N'oubliez surtout pas d'inclure la librairie de math !}

    \subsection{fractal}

    
    La \textbf{turtle} est une librarie Python pré-installée qui permet aux utilisateur de dessiner différente forme sur une toile blanche.

    \Terminal{
    \\
    import turtle\\

    pen = turtle.Turtle()\\
    screen = turtle.Screen()\\
    \\
    t.forward(50)\\
    t.up()\\
    t.forward(50)\\
    t.down()\\
    t.forward(50)
    }

    \Warning{
    Ce code est un exemple tres simpliste, il vous montre seulement quelque fonction de \textbf{turtle}. 
    En plus on a même pas le temps de voir le résultat...
    }

    \Hint{Déjà entendu parler d'une mainloop ?}

    \ImageCenter{assets/tree.png}{200}
}

\end{document}