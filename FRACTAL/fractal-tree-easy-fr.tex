\documentclass[a4paper, 12pt]{article}

\usepackage{/template/configs/base}

\begin{document}

\Init{Les fractales}{La poupée russe des mathématiques}{}{1}
\InitInfo{fractalTree}{Python 3}{turtle}{}{2}

\Hypersetup

\Cover
\Intro{
    Voilà le jour du grand bain approche !\\
    Mais avant il va falloir passer par le pédiluve et se tremper les pieds. 
    Vous allez découvrir à travers ce projet un concept mathématique se nommant les fractales. 
    En zoomant sur une partie de la figure, il est possible de retrouver toute la figure à n'importe quel endroit. 
    Aujourd'hui vous allez voir une de ces formes: les \emph{ARBRES FRACTALES}.\\
    En utilisant Python et grâce à l'aide d'une \textbf{turtle} vous allez apprendre à dessiner ces arbres fractales.
    \emph{\textbf{C'est pas génial ça ?}}
}

\section{Qu'es ce qu'une turtle ?}
    La \textbf{turtle} est une librarie Python pré-installée qui permet aux utilisateur de dessiner différente forme sur une toile blanche.

\subsection{Comment l'utiliser ?}
\Terminal{
    \\
    import turtle\\

    pen = turtle.Turtle()\\
    screen = turtle.Screen()\\
    \\
    t.forward(50)\\
    t.up()\\
    t.forward(50)\\
    t.down()\\
    t.forward(50)
}

\Warning{
    Ce code est un exemple tres simpliste, il vous montre seulement quelque fonction de \textbf{turtle}. 
    En plus on a même pas le temps de voir le résultat...
}
\Hint{Déjà entendu parler d'une mainloop ?}

\section{Résultat Attendu}
\ImageCenter{assets/tree.png}{250}

\end{document}
