\documentclass[a4paper, 12pt]{article}

\usepackage{/template/configs/base}

\begin{document}

\Init{Le générateur de cercles}{PacMan en sueur}{Python}{3}
\InitInfo{}{Python 3}{}{}{7}

\Hypersetup

\Cover

\Intro{
    Voilà le jour du grand bain approche ! \\
    Mais avant il va falloir passer par la pédiluve.\\
    Le but du projet est de réaliser un générateur de cercles avec différentes tailles données en paramètres.
}

\section{Résultat Attendu}
\ImageCenter{assets/one.png}{100}
\ImageCenter{assets/two.png}{300}

Il sera important de tester le binaire fournis afin de comprendre la logique.

\section{Fonction}

\Warning{La fonction suivante vous sera demandée !}
\Terminal{def printcircle(radius):}
Cette fonction devra calculer et afficher chaques parties du cercle.
\Warning{N'oubliez surtout pas d'inclure la librairie de math !}

\end{document}
