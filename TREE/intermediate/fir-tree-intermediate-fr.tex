\documentclass[a4paper, 12pt]{article}

\usepackage{/template/configs/base}

\begin{document}

\Init{Le générateur d'arbres}{L'ami de GreanPeace}{Python}{3}
\InitInfo{}{Python 3}{}{}{7}

\Hypersetup

\Cover

\Intro{
    Voilà le jour du grand bain approche ! \\
    Mais avant il va falloir passer par la pédiluve.\\
    Le but du projet est de réaliser un arbre avec différents niveaux donnéees en paramètres.
}

\section{Résultat Attendu}
\ImageCenter{assets/one.png}{100}
\ImageCenter{assets/two.png}{300}

Il sera important pour réaliser ce programme de séparer de façon logique vos fonctions.

\section{Fonctions}

\Warning{Les fonctions suivantes vous seront demandées !}
\Terminal{def trunck(size):}
Cette fonction devra servir à calculer et afficher le tronc de l'arbre.
\Terminal{def leaves():}
Cette fonction permet le calcul et l'affichage complet des feuiles !
\Warning{La fonction sera appellée plusieurs fois par votre programme.}

\end{document}
